\section{Bedienung}

\subsection{Allgemeines zum \textit{MAX274}}
Die integrierte Schaltung (IC) \textit{MAX274} ist in der Lage aktive Tief- und Bandpassfilter gerader Ordnung umzusetzen. Hierbei verwendet werden können Butterworth-, Tschebyscheff- und Besselfilter. 
Besonderheit des IC ist hierbei, dass sich die Filtercharakteristiken allein durch die Beschaltung mit Widerständen einstellen lassen.
Ein \textit{MAX274} enthält dabei vier Filtersektionen zweiter Ordnung. 
Durch Kaskadierung dieser Sektionen lassen sich beliebig hohe gerade Filterordnungen implementieren.
\\ \newline
Die hier beschriebene Software erlaubt den Entwurf von Filtern und die anschließende Berechnung von Widerstandswerten zur Beschaltung des IC.

Achtung: Die hier beschriebene Software ermöglicht es außerdem Filter ungerader Ordnung, Hochpassfilter und Filter nach Cauerapproximation zu designen. Diese können \underline{nicht} auf dem \textit{MAX274} realisiert werden!
Knöpfe zur Berechnung der Widerstandswerte werden in diesem Fall ausgeblendet.

\subsection{Entwurf von Filtern}

Zunächst sollte im Feld \textit{Filtertyp} die gewünschte Filterart ausgewählt werden.
Es kann zwischen Tief-, Hoch- und Bandpassfiltern gewählt werden.

Darunter lassen sich die maximale Dämpfung im Durchlassbereich ($A_d$) und die minimale Dämpfung im Sperrbereich ($A_s$) des Filters, sowie seine Grenzfrequenzen angeben. 
Die Eingabefelder der nicht ausgewählten Filtertypen sind grau hinterlegt.

Hieraufhin werden die minimalen Filterordnungen der verschiedenen Approximationen berechnet, die zur Einhaltung der gewünschten Tolleranzvorgaben benötigt werden.
Die Software unterstützt das Berechnen von Filtern bis zur Ordnung 24.

Achtung: Es können nur Widerstandswerte für Filter gerader Ordnung berechnet werden.
Ist z.\,B. eine minimale Filterordnung von 3 errechnet worden, sollte diese auf mindestens 4 erhöht werden.

Nach Auswahl der Filterapproximation können durch das Drücken des Knopfs \textit{Filter berechnen} die Pol- und Nullstellen, sowie $f_0$ und $Q$ der Filtersektionen errechnet werden.

Wird der Knopf \textit{Plot} betätigt wird die Übertragungsfunktion des zuletzt berechneten Filters als Plot dargestellt. 
Hier werden Betrag, Phase und Gruppenlaufzeit aufgeführt.

Entspricht das Filter dem gewünschten Verhalten kann dieses durch den Knopf \textit{Exportieren} in den Tab zur Widerstandsberechnung exportiert werden.

\subsection{Bestimmung von Widerstandswerten}

Nach dem Exportieren einer Filtercharakteristik werden automatisch die Widerstandswerte $R_1, R_2, R_3$ und $R_4$ zur Beschaltung des \textit{MAX274} berechnet und nach Sektionen sortiert in einer Tabelle aufgeführt.
Die Berechnung der Widerstandswerte aus $f_0$ und $Q$ lässt sich außerdem jederzeit durch das betätigen des Knopfs \textit{Berechne Widerstandswerte} manuell ausführen.

Oftmals gibt es Vorgaben an die Auswahl von Widerstandswerten, wie z.\,B. die E-Reihen. Die Widerstandswerte lassen sich in der Tabelle ändern. 
Daraufhin können mit \textit{Berechne Pol-/Nullstellen aus Widerstandswerten} neue Pol- und Nullstellen, sowie $f_0$s und $Q$s berechnet werden.

Das modifiezierte Übertragungsverhalten lässt sich über das Betätigen von \textit{Plot} anzeigen.
Über das daneben liegende Drop-Down-Menü kann auch das Übertragungsverhalten einzelner Filtersektionen darstellen.
